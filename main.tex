\documentclass{article}

\usepackage{amsmath,amssymb,amsthm}

\setlength{\oddsidemargin}{0.25 in}
\setlength{\evensidemargin}{-0.25 in}
\setlength{\topmargin}{-0.6 in}
\setlength{\textwidth}{6.5 in}
\setlength{\textheight}{8.5 in}
\setlength{\headsep}{0.75 in}
\setlength{\parindent}{0 in}
\setlength{\parskip}{0.1 in}

\newtheorem{theorem}{Theorem}
\newtheorem{corollary}{Corollary}
\newtheorem{proposition}{Proposition}
\newtheorem*{remark}{Remark}
\theoremstyle{definition}
\newtheorem{example}{Example}
\newtheorem{definition}{Definition}

\newcommand{\lecture}[4]{
   \pagestyle{myheadings}
   \thispagestyle{plain}
   \newpage
%   \setcounter{lecnum}{#1}
   \setcounter{page}{1}
   \noindent
   \begin{center}
   \framebox{
      \vbox{\vspace{2mm}
    \hbox to 6.58in { {\bf CSC~565: Graph Theory
                        \hfill North Carolina State University} }
    \hbox to 6.58in { {\bf Fall 2019
                        \hfill Computer Science} }
       \vspace{4mm}
       \hbox to 6.28in { {\Large \hfill Lecture #1: #2  \hfill} }
       \vspace{2mm}
       \hbox to 6.28in { {\it Lecturer: {\it Don Sheehy {\tt <drsheehy@ncsu.edu>}} \hfill Scribe: #4} }
      \vspace{2mm}}
   }
   \end{center}
   \markboth{Lecture #1: #2}{Lecture #1: #2}
   \vspace*{4mm}
}
\usepackage{graphics,tikz}
\usepackage{chngcntr}

%\newtheoremstyle{case}{}{}{}{}{\bfseries}{:}{ }{}
%\theoremstyle{case}
\newtheorem{case}{Case}
\counterwithin*{case}{theorem}
%\def\geom{\text{geom}}
%\def\sim{\text{sim}}
%\def\R{\mathbb{R}}

\begin{document}

%FILL IN THE RIGHT INFO.
%\lecture{**LECTURE-NUMBER**}{**DATE**}{**LECTURER**}{**SCRIBE**}
\lecture{7}{Sep 16, 2019}{Don Sheehy}{Palash Gupta, Rishabh Agrawal, Jash Dhakad }

\section{Menger's Theorem}
\begin{theorem}[Menger's Theorem]
    A graph G is k-connected iff for all u,v $\in V_{G}$, there exist k disjoint\footnote{Disjoint paths are those which no shared vertices except the end vertices} paths from u to v.
\end{theorem}

\section{Trees and forests}
\begin{theorem}
    G is bipartite iff it contains no odd length cycles.
\end{theorem}

\begin{proof}
    To prove this we need to prove both the cases, i.e. if a graph is bipartite, it contains no odd cycles and if a graph contains no odd cycles, it is bipartite.\\

    \begin{case}
        If a graph is bipartite, it cannot contain an odd cycle.\\
        \\
        Let $G$ be a bipartite graph. We'll prove that it contains no odd cycles by contradiction. First let us assume that it contains an odd cycle of size $n$. We'll number all the vertices on this cycle from $0$ to $n-1$ such that there is an edge between $i$ and $i-1$ for all $i$ from $1$ to $n-1$. Also, there is an edge between $n-1$ and $0$, completing the cycle. As this is a bipartite graph, all the vertices can be divided into two sets $A$ and $B$, such that for any edge $(u,v) \in E_{G}$, $u \in A$ and $v \in B$. Let's say vertex $0 \in A$, then $1 \in B$ (because there is an edge between $0$ and $1$), $2 \in A$ and so on. We can observe that all even numbered vertices belong to $A$ whereas all odd numbered vertices belong to $B$. $n$ is an odd number (as this is an odd cycle), therefore $n-1$ is an even number and vertex labeled $n-1 \in A$. But there is an edge between $n-1$ and $0$ and both belong to set $A$. This is not allowed in a bipartite graph. Therefore, our assumption must be wrong and there cannot exist an odd length cycle in graph $G$.
    \end{case}

    \begin{case}
        If a graph has no odd length cycle, it is bipartite.\\
        \\
        Let $G$ be a graph with no odd length cycle. We need to prove that it is bipartite. We'll do that by defining an algorithm to divide the vertices in two sets and then prove that conditions of a bipartite graph are satisfied.\\
        \\Following are the steps to partition the vertices into sets $A$ and $B$:
        \begin{enumerate}
            \item Choose any arbitrary vertex as the starting point. Let's call it S.
            \item Label all vertices by the minimum number of edges we need to travel to S, i.e. by the minimum distance from S.
            \item From the vertices which are yet to be labelled (as they are not connected to S), choose an arbitrary vertex and repeat the process till all vertices are labelled.
            \item Put all vertices with even labels in set $A$ and all vertices with odd labels in set $B$.
        \end{enumerate}
        Now we need to show that sets $A$ and $B$ so formed follow the conditions of a bipartite graph, i.e. no edge can exist between two vertices of same set (either $A$ or $B$), i.e. that $A$ and $B$ are independent sets.
    \end{case}
\end{proof}
  
\begin{definition}
    A forest is a graph that contains no cycles.
\end{definition}
\begin{definition}
    A connected forest is called a tree.
\end{definition}
\begin{theorem}
    A connected graph is a tree iff it contains no {$C_3$} minor.
\end{theorem}

\begin{proof}
    \begin{case}
To prove that if the graph is a tree then it does not contain a $C_3$ minor.\\
\\
Proof by contradiction: \\
Let us assume that the graph G contains a $C_3$ minor. If this is true, we can state that the subgraph of G can be reduced to a $C_3$ minor by edge contraction, which means that there exists a cycle in the graph. The definition of tree states that any graph which does not have any cycle. Hence, our assumption was wrong and we can say that G is a tree if it does not contain a $C_3$ minor.  \\
    \end{case}

    \begin{case}
To prove that if the graph does not contain a $C_3$ minor then it is a tree.\\
\\Proof by contradiction: 
\\
Let us assume that the graph, G is not a tree which means there exists a connected graph with a cycle. If this is true, then we can say that any cycle can be reduced to a $C_3$ minor by edge contraction which contradicts the assumption made and suggests that the graph contains a $C_3$ minor. Hence, our assumption was wrong and so we can say that if the graph does not contain a $C_3$ minor then it is a tree.   


\begin{center}
    
\tikzset{every picture/.style={line width=0.75pt}} %set default line width to 0.75pt        

\begin{tikzpicture}[x=0.75pt,y=0.75pt,yscale=-1,xscale=1]
%uncomment if require: \path (0,723); %set diagram left start at 0, and has height of 723

%Straight Lines [id:da17346525375595823] 
\draw    (169,65) -- (209,109) ;


%Straight Lines [id:da642015983522537] 
\draw    (465,63) -- (527,121) ;


%Straight Lines [id:da14931962569626012] 
\draw    (169,65) -- (149,120) ;


%Straight Lines [id:da08080995923232726] 
\draw    (169,65) -- (259,77) ;


%Straight Lines [id:da7297042825487361] 
\draw    (406,106) -- (446,150) ;


%Straight Lines [id:da9983591846519424] 
\draw    (249,153) -- (289,197) ;


%Straight Lines [id:da39921401012667346] 
\draw    (259,77) -- (406,106) ;


%Straight Lines [id:da8884551136199939] 
\draw    (465,63) -- (406,106) ;


%Straight Lines [id:da36258657409434103] 
\draw    (149,120) -- (140,66) ;


%Straight Lines [id:da8653085358128052] 
\draw    (209,109) -- (249,153) ;


%Straight Lines [id:da9820821296870912] 
\draw    (289,197) -- (381,169) ;


%Straight Lines [id:da16410625947921065] 
\draw    (289,197) -- (373,208) ;


%Straight Lines [id:da38435752905494924] 
\draw    (446,150) -- (381,169) ;


%Straight Lines [id:da3382749363158353] 
\draw    (289,197) -- (329,241) ;


%Straight Lines [id:da9797038377585146] 
\draw    (289,197) -- (254,248) ;


%Shape: Circle [id:dp47392873976328787] 
\draw   (105,102) .. controls (105,61.68) and (137.68,29) .. (178,29) .. controls (218.32,29) and (251,61.68) .. (251,102) .. controls (251,142.32) and (218.32,175) .. (178,175) .. controls (137.68,175) and (105,142.32) .. (105,102) -- cycle ;
%Shape: Circle [id:dp971042509290946] 
\draw   (377,125) .. controls (377,80.26) and (413.26,44) .. (458,44) .. controls (502.74,44) and (539,80.26) .. (539,125) .. controls (539,169.74) and (502.74,206) .. (458,206) .. controls (413.26,206) and (377,169.74) .. (377,125) -- cycle ;
%Left Arrow [id:dp3157799490145934] 
\draw   (303.39,410) -- (290.1,367.53) -- (296.85,367.57) -- (297.16,303.97) -- (310.66,304.03) -- (310.35,367.63) -- (317.1,367.66) -- cycle ;
%Straight Lines [id:da5554865929934945] 
\draw    (259,584) -- (298,641) ;


%Straight Lines [id:da012688686285775796] 
\draw    (145,452) -- (161,515) ;


%Straight Lines [id:da4841755849594085] 
\draw    (212,639) -- (259,584) ;


%Straight Lines [id:da8326150314982221] 
\draw    (161,515) -- (197,441) ;


%Straight Lines [id:da10427054510928668] 
\draw    (197,441) -- (259,584) ;


%Straight Lines [id:da9955678761970332] 
\draw    (392,483) -- (259,584) ;


%Straight Lines [id:da24379949366012799] 
\draw    (197,441) -- (392,483) ;


%Straight Lines [id:da8190446038982127] 
\draw    (259,584) -- (359,605) ;


%Straight Lines [id:da8459756464497579] 
\draw    (392,483) -- (471,440) ;


%Straight Lines [id:da48257728570050773] 
\draw    (471,440) -- (511,484) ;



% Text Node
\draw (392,348) node  [align=left] {Edge Contraction};
% Text Node
\draw (476,601) node  [align=left] {The subgraph has a cycle};


\end{tikzpicture}

\end{center}

    \end{case}
\end{proof}

\begin{text}
{\textbf{Claim:} Given a tree and two vertices, u,v $\in V_{T}$ there exists a unique path u to v in T.}   
\end{text}

\begin{text}
{\textbf {Question: }Prove that no tree is 2 connected.}
    \begin{proof}
        Bla Bla Bla....       
    \end{proof}
\end{text}
\begin{text}
{\textbf {Corollary: }Removal of an edge in a tree will separate the graph.}
    \begin{proof}
        Bla Bla Bla....       
    \end{proof}
\end{text}
\begin{definition}
A leaf \footnote{A \textbf{star} has (n-1) leaves and a \textbf{path} has 2 leaves} of a tree (graph) is a degree 1 vertex. On separating leaf we are still left with a connected graph.
\end{definition}

\begin{theorem}
    Every tree with at least 2 vertices has at least 2 leaves.
    \begin{proof}[Proof idea:]
        Path is a minor of a tree. \\
    \end{proof}
\end{theorem}
\begin{text}
{\textbf {Question: }Prove every forest is bipartite}
    \begin{proof}
        Bla Bla Bla....       
    \end{proof}
\end{text}
\begin{text}
{\textbf {Question: }How many edges can a tree with n vertices have?}
    \begin{proof}[Solution: ]
        A tree with n vertices has (n-1) edges.       
    \end{proof}
    {\textbf {Note: }A forest will have at most (n-1) edges.}
    
\end{text}

\end{document}
