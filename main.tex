\documentclass{article}

\usepackage{amsmath,amssymb,amsthm}

\setlength{\oddsidemargin}{0.25 in}
\setlength{\evensidemargin}{-0.25 in}
\setlength{\topmargin}{-0.6 in}
\setlength{\textwidth}{6.5 in}
\setlength{\textheight}{8.5 in}
\setlength{\headsep}{0.75 in}
\setlength{\parindent}{0 in}
\setlength{\parskip}{0.1 in}

\newtheorem{theorem}{Theorem}
\newtheorem{corollary}{Corollary}
\newtheorem{proposition}{Proposition}
\newtheorem*{remark}{Remark}
\theoremstyle{definition}
\newtheorem{example}{Example}
\newtheorem{definition}{Definition}

\newcommand{\lecture}[4]{
   \pagestyle{myheadings}
   \thispagestyle{plain}
   \newpage
%   \setcounter{lecnum}{#1}
   \setcounter{page}{1}
   \noindent
   \begin{center}
   \framebox{
      \vbox{\vspace{2mm}
    \hbox to 6.58in { {\bf CSC~565: Graph Theory
                        \hfill North Carolina State University} }
    \hbox to 6.58in { {\bf Fall 2019
                        \hfill Computer Science} }
       \vspace{4mm}
       \hbox to 6.28in { {\Large \hfill Lecture #1: #2  \hfill} }
       \vspace{2mm}
       \hbox to 6.28in { {\it Lecturer: {\it Don Sheehy {\tt <drsheehy@ncsu.edu>}} \hfill Scribe: #4} }
      \vspace{2mm}}
   }
   \end{center}
   \markboth{Lecture #1: #2}{Lecture #1: #2}
   \vspace*{4mm}
}
\usepackage{graphics,tikz}
\def\geom{\text{geom}}
\def\sim{\text{sim}}
\def\R{\mathbb{R}}\begin{document}

%FILL IN THE RIGHT INFO.
%\lecture{**LECTURE-NUMBER**}{**DATE**}{**LECTURER**}{**SCRIBE**}
\lecture{7}{Sep 16, 2019}{Don Sheehy}{Palash Gupta, Rishabh Agrawal, Jash Dhakad }

  % \title{Lecture 6}
  % \author{Scribed by: }
  % \maketitle
\section{Menger's Theorem}
\begin{definition}
G is k connected iff for all u,v $\in V_{G}$ there exist k disjoint paths from u to v.\\
Disjoint suggests that there will be no shared vertices except ends.

%Let X and Y be simplicial complexes. A simplicial map $f: X \rightarrow Y$ is a pair of functions:
%\centerline {$f_v: V_X \rightarrow V_Y$ and $f_S: S_X \rightarrow S_Y$}

%where $f_S(\sigma) = \{ f_V(u): u \in \sigma \}$
\end{definition}
\begin{theorem}
    G is bipartite iff it contains no odd length cycles.
\end{theorem}
\begin{proof}[Proof idea]
  
\end{proof}
  
\begin{definition}
    A forest is a graph that contains no cycles.
\end{definition}
\begin{definition}
    A connected forest is called a tree.
\end{definition}
\begin{theorem}
    A connected graph is a tree iff it contains no {$C_3$} minor.
\end{theorem}
\begin{proof}[Proof idea:]
    Edge contraction. \\
    
\end{proof}
Based on the preimages of the {$C_3$} minor we can backtrack to where this might have originated from.

\begin{text}
{\textbf{Claim:} Given a tree and two vertices, u,v $\in V_{T}$ there exists a unique path u to v in T.}   
\end{text}

\begin{text}
{\textbf {Question: }Prove that no tree is 2 connected.}
    \begin{proof}
        Bla Bla Bla....       
    \end{proof}
\end{text}
\begin{text}
{\textbf {Corollary: }Removal of an edge in a tree will separate the graph.}
    \begin{proof}
        Bla Bla Bla....       
    \end{proof}
\end{text}
\begin{definition}
A leaf \footnote{A \textbf{star} has (n-1) leaves and a \textbf{path} has 2 leaves} of a tree (graph) is a degree 1 vertex. On separating leaf we are still left with a connected graph.
\end{definition}

\begin{theorem}
    Every tree with atleast 2 vertices has atleast 2 leaves.
    \begin{proof}[Proof idea:]
        Path is a minor of a tree. \\
    \end{proof}
\end{theorem}
\begin{text}
{\textbf {Question: }Prove every forest is bipartite}
    \begin{proof}
        Bla Bla Bla....       
    \end{proof}
\end{text}
\begin{text}
{\textbf {Question: }How many edges can a tree with n vertices have?}
    \begin{proof}[Solution: ]
        A tree with n vertices has (n-1) edges.       
    \end{proof}
    {\textbf {Note: }A forest will have atmost (n-1) edges.}
    
\end{text}

\end{document}