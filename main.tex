\documentclass{article}

\usepackage{amsmath,amssymb,amsthm}

\input{macros}
\usepackage{graphics,tikz}
\usepackage{inputenc}
\def\geom{\text{geom}}
\def\sim{\text{sim}}
\def\R{\mathbb{R}}\begin{document}

%FILL IN THE RIGHT INFO.
%\lecture{**LECTURE-NUMBER**}{**DATE**}{**LECTURER**}{**SCRIBE**}
\lecture{7}{Sep 16, 2019}{Don Sheehy}{Palash Gupta, Rishabh Agrawal, Jash Dhakad }

  % \title{Lecture 6}
  % \author{Scribed by: }
  % \maketitle
\section{Menger's Theorem}
\begin{theorem}
G is k connected iff for all u,v $\in V_{G}$ there exist k disjoint paths from u to v.\\
(Disjoint suggests that there will be no shared vertices except ends.)

%Let X and Y be simplicial complexes. A simplicial map $f: X \rightarrow Y$ is a pair of functions:
%\centerline {$f_v: V_X \rightarrow V_Y$ and $f_S: S_X \rightarrow S_Y$}

%where $f_S(\sigma) = \{ f_V(u): u \in \sigma \}$
\end{theorem}
\begin{theorem}
    G is bipartite iff it contains no odd length cycles.
\end{theorem}
\begin{proof}[Proof idea]
  
\end{proof}
  
\begin{definition}
    A forest is a graph that contains no cycles.
\end{definition}
\begin{definition}
    A connected forest is called a tree.
\end{definition}
\begin{theorem}
    A connected graph is a tree iff it contains no {$C_3$} minor.
\end{theorem}
\begin{proof}
Case 1: To prove that if the graph is a tree then it does not contain a $C_3$ minor.\\
\\
Proof by contradiction: \\
Let us assume that the graph G contains a $C_3$ minor. If this is true, we can state that the subgraph of G can be reduced to a $C_3$ minor by edge contraction, which means that there exists a cycle in the graph. The definition of tree states that any graph which does not have any cycle. Hence, our assumption was wrong and we can say that G is a tree if it does not contain a $C_3$ minor.  \\

Case 2: To prove that if the graph does not contain a $C_3$ minor then it is a tree.\\
\\Proof by contradiction: 
\\
Let us assume that the graph, G is not a tree which means there exists a connected graph with a cycle. If this is true, then we can say that any cycle can be reduced to a $C_3$ minor by edge contraction which contradicts the assumption made and suggests that the graph contains a $C_3$ minor. Hence, our assumption was wrong and so we can say that if the graph does not contain a $C_3$ minor then it is a tree.   


\begin{center}
    
\tikzset{every picture/.style={line width=0.75pt}} %set default line width to 0.75pt        

\begin{tikzpicture}[x=0.75pt,y=0.75pt,yscale=-1,xscale=1]
%uncomment if require: \path (0,723); %set diagram left start at 0, and has height of 723

%Straight Lines [id:da17346525375595823] 
\draw    (169,65) -- (209,109) ;


%Straight Lines [id:da642015983522537] 
\draw    (465,63) -- (527,121) ;


%Straight Lines [id:da14931962569626012] 
\draw    (169,65) -- (149,120) ;


%Straight Lines [id:da08080995923232726] 
\draw    (169,65) -- (259,77) ;


%Straight Lines [id:da7297042825487361] 
\draw    (406,106) -- (446,150) ;


%Straight Lines [id:da9983591846519424] 
\draw    (249,153) -- (289,197) ;


%Straight Lines [id:da39921401012667346] 
\draw    (259,77) -- (406,106) ;


%Straight Lines [id:da8884551136199939] 
\draw    (465,63) -- (406,106) ;


%Straight Lines [id:da36258657409434103] 
\draw    (149,120) -- (140,66) ;


%Straight Lines [id:da8653085358128052] 
\draw    (209,109) -- (249,153) ;


%Straight Lines [id:da9820821296870912] 
\draw    (289,197) -- (381,169) ;


%Straight Lines [id:da16410625947921065] 
\draw    (289,197) -- (373,208) ;


%Straight Lines [id:da38435752905494924] 
\draw    (446,150) -- (381,169) ;


%Straight Lines [id:da3382749363158353] 
\draw    (289,197) -- (329,241) ;


%Straight Lines [id:da9797038377585146] 
\draw    (289,197) -- (254,248) ;


%Shape: Circle [id:dp47392873976328787] 
\draw   (105,102) .. controls (105,61.68) and (137.68,29) .. (178,29) .. controls (218.32,29) and (251,61.68) .. (251,102) .. controls (251,142.32) and (218.32,175) .. (178,175) .. controls (137.68,175) and (105,142.32) .. (105,102) -- cycle ;
%Shape: Circle [id:dp971042509290946] 
\draw   (377,125) .. controls (377,80.26) and (413.26,44) .. (458,44) .. controls (502.74,44) and (539,80.26) .. (539,125) .. controls (539,169.74) and (502.74,206) .. (458,206) .. controls (413.26,206) and (377,169.74) .. (377,125) -- cycle ;
%Left Arrow [id:dp3157799490145934] 
\draw   (303.39,410) -- (290.1,367.53) -- (296.85,367.57) -- (297.16,303.97) -- (310.66,304.03) -- (310.35,367.63) -- (317.1,367.66) -- cycle ;
%Straight Lines [id:da5554865929934945] 
\draw    (259,584) -- (298,641) ;


%Straight Lines [id:da012688686285775796] 
\draw    (145,452) -- (161,515) ;


%Straight Lines [id:da4841755849594085] 
\draw    (212,639) -- (259,584) ;


%Straight Lines [id:da8326150314982221] 
\draw    (161,515) -- (197,441) ;


%Straight Lines [id:da10427054510928668] 
\draw    (197,441) -- (259,584) ;


%Straight Lines [id:da9955678761970332] 
\draw    (392,483) -- (259,584) ;


%Straight Lines [id:da24379949366012799] 
\draw    (197,441) -- (392,483) ;


%Straight Lines [id:da8190446038982127] 
\draw    (259,584) -- (359,605) ;


%Straight Lines [id:da8459756464497579] 
\draw    (392,483) -- (471,440) ;


%Straight Lines [id:da48257728570050773] 
\draw    (471,440) -- (511,484) ;



% Text Node
\draw (392,348) node  [align=left] {Edge Contraction};
% Text Node
\draw (476,601) node  [align=left] {The subgraph has a cycle};


\end{tikzpicture}

\end{center}

\end{proof}

\begin{text}
{\textbf{Claim:} Given a tree and two vertices, u,v $\in V_{T}$ there exists a unique path u to v in T.}   
\end{text}

\begin{text}
{\textbf {Question: }Prove that no tree is 2 connected.}
    \begin{proof}
        Bla Bla Bla....       
    \end{proof}
\end{text}
\begin{text}
{\textbf {Corollary: }Removal of an edge in a tree will separate the graph.}
    \begin{proof}
        Bla Bla Bla....       
    \end{proof}
\end{text}
\begin{definition}
A leaf \footnote{A \textbf{star} has (n-1) leaves and a \textbf{path} has 2 leaves} of a tree (graph) is a degree 1 vertex. On separating leaf we are still left with a connected graph.
\end{definition}

\begin{theorem}
    Every tree with atleast 2 vertices has atleast 2 leaves.
    \begin{proof}[Proof idea:]
        Path is a minor of a tree. \\
    \end{proof}
\end{theorem}
\begin{text}
{\textbf {Question: }Prove every forest is bipartite}
    \begin{proof}
        Bla Bla Bla....       
    \end{proof}
\end{text}
\begin{text}
{\textbf {Question: }How many edges can a tree with n vertices have?}
    \begin{proof}[Solution: ]
        A tree with n vertices has (n-1) edges.       
    \end{proof}
    {\textbf {Note: }A forest will have atmost (n-1) edges.}
    
\end{text}

\end{document}